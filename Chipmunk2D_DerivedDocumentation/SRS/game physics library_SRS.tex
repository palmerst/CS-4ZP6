\documentclass[12pt]{article}

\usepackage{bm}
\usepackage{amsmath, mathtools}
\usepackage{amsfonts}
\usepackage{amssymb}
\usepackage{graphicx}
\usepackage{colortbl}
\usepackage{xr}
\usepackage{hyperref}
\usepackage{longtable}
\usepackage{xfrac}
\usepackage{tabularx}
\usepackage{float}
\usepackage{siunitx}
\usepackage{booktabs}

%\usepackage{refcheck}

\hypersetup{
    bookmarks=true,         % show bookmarks bar?
      colorlinks=true,       % false: boxed links; true: colored links
    linkcolor=red,          % color of internal links (change box color with linkbordercolor)
    citecolor=green,        % color of links to bibliography
    filecolor=magenta,      % color of file links
    urlcolor=cyan           % color of external links
}
\newcommand{\NN}[1]{{\color{red}#1}}
\newcommand{\WSS}[1]{{\color{blue}#1}}

\newcommand{\colZwidth}{1.0\textwidth}
\newcommand{\blt}{- } %used for bullets in a list
\newcommand{\colAwidth}{0.13\textwidth}
\newcommand{\colBwidth}{0.82\textwidth}
\newcommand{\colCwidth}{0.1\textwidth}
\newcommand{\colDwidth}{0.05\textwidth}
\newcommand{\colEwidth}{0.8\textwidth}
\newcommand{\colFwidth}{0.17\textwidth}
\newcommand{\colGwidth}{0.5\textwidth}
\newcommand{\colHwidth}{0.28\textwidth}
\newcounter{defnum} %Definition Number
\newcommand{\dthedefnum}{GD\thedefnum}
\newcommand{\dref}[1]{GD\ref{#1}}
\newcounter{datadefnum} %Datadefinition Number
\newcommand{\ddthedatadefnum}{DD\thedatadefnum}
\newcommand{\ddref}[1]{DD\ref{#1}}
\newcounter{theorynum} %Theory Number
\newcommand{\tthetheorynum}{T\thetheorynum}
\newcommand{\tref}[1]{T\ref{#1}}
\newcounter{tablenum} %Table Number
\newcommand{\tbthetablenum}{T\thetablenum}
\newcommand{\tbref}[1]{TB\ref{#1}}
\newcounter{assumpnum} %Assumption Number
\newcommand{\atheassumpnum}{P\theassumpnum}
\newcommand{\aref}[1]{A\ref{#1}}
\newcounter{goalnum} %Goal Number
\newcommand{\gthegoalnum}{P\thegoalnum}
\newcommand{\gsref}[1]{GS\ref{#1}}
\newcounter{instnum} %Instance Number
\newcommand{\itheinstnum}{IM\theinstnum}
\newcommand{\iref}[1]{IM\ref{#1}}
\newcounter{reqnum} %Requirement Number
\newcommand{\rthereqnum}{P\thereqnum}
\newcommand{\rref}[1]{R\ref{#1}}
\newcounter{lcnum} %Likely change number
\newcommand{\lthelcnum}{LC\thelcnum}
\newcommand{\lcref}[1]{LC\ref{#1}}

\newcommand{\tclad}{T_\text{CL}}
\newcommand{\degree}{\ensuremath{^\circ}}
\newcommand{\progname}{SWHS}


\usepackage{fullpage}

\begin{document}

\title{Software Requirements Specification for Chipmunk2D} 
\author{Alex Halliwushka}
\date{\today}
	
\maketitle

\tableofcontents

%%%%%%%%%%%%%%%%%%%%%%%%
%
%	1.) REFERENCE MATERIAL 
%
%%%%%%%%%%%%%%%%%%%%%%%%

\section{Reference Material}

This section records information for easy reference.


%1.1 Table of units
\subsection{Table of Units}

Throughout this document SI (Syst\`{e}me International d'Unit\'{e}s) is employed
as the unit system. For each unit, the symbol is given followed by a
description of the unit with the SI name.
~\newline

\renewcommand{\arraystretch}{1.2}

  \noindent \begin{tabular}{l l l} 
    \toprule		
    \textbf{symbol} & \textbf{unit} & \textbf{SI}\\
    \midrule 
    \si{\meter} & length & meter\\
    \si{\kilogram} & mass	& kilogram\\
    \si{\second} & time & second\\
    \si{\newton} & force & Newton\\
    \si{\radian} &   angle & radians\\
    \bottomrule
  \end{tabular}


%1.2 Table of units
\subsection{Table of Symbols}

The table that follows summarizes the symbols used in this document along with
their units. The symbols are listed in alphabetical order.

\renewcommand{\arraystretch}{1.2}
%\noindent \begin{tabularx}{1.0\textwidth}{l l X}
\noindent \begin{longtable}{l l p{12cm}} \toprule
  \textbf{symbol} & \textbf{unit} & \textbf{description}\\
  \midrule 

  $a$ & \si[per-mode=symbol] {\metre\per\second^{2}} & acceleration
  \\
  $\alpha$ & \si[per-mode=symbol] {\radian\per\second^{2}} & angular acceleration
  \\
  $C_\text{R}$ &unitless & Coefficient of restitution 
  \\ 
  $d$ & \si[per-mode=symbol] {\metre}   & Displacement
  \\ 
  $d_\text{p}$ & \si[per-mode=symbol] {\metre}   & Distance between particle and the axis of rotation
  \\ 
  $\bf{F}$ & \si[per-mode=symbol] {\newton}   & Force
  \\ 
  $g$ & \si[per-mode=symbol] {\newton}   & Acceleration due to gravity
  \\ 
  $G$ & \si[per-mode=symbol] {\metre^{3}\per\kilo\gram \second^{2}}   & Gravitational constant  ($6.673*10^-{11})$
  \\
  $\bf{I}$ & \si[per-mode=symbol] {\kilo\gram\metre^{2}}   & Moment of inertia
  \\ 
  $\bf{J}$ & \si[per-mode=symbol] {\newton\second}   & Impulse
  \\ 
  $k$ & \si[per-mode=symbol] {\newton\per\metre}   & Spring constant
  \\ 
  $L$ & \si[per-mode=symbol] {\metre}   & Length
  \\ 
  $m$ & \si[per-mode=symbol] {\kilo\gram}   & Mass
  \\ 
  $\bf{\omega}$ & \si[per-mode=symbol] {\radian\per\second}   & Angular velocity
  \\ 
  $\bf{p_\text{i}}$ & \si[per-mode=symbol] {\metre}  & Initial position
  \\ 
  $\bf{p_\text{f}}$ & \si[per-mode=symbol] {\metre}   & Final position
  \\ 
  $\bf{p_\text{com}}$ & \si[per-mode=symbol] {\metre}   & Position of the center of mass
  \\ 
  $\bf{p_\text{force}}$ & \si[per-mode=symbol] {\metre}   & Position the force vector is acting on the object
  \\ 
  $\phi$ & \si[per-mode=symbol] {\radian}   & Orientation
  \\ 
  $r$ & \si[per-mode=symbol] {\metre}   &Distance between the force and the axis of rotation
  \\ 
  $\rho$ & \si[per-mode=symbol] {\kilo\gram\per\metre^{3}}   &Density
  \\ 
  $t$ & \si[per-mode=symbol] {\second}   &Time
  \\ 
  $\bf{\tau}$ & \si[per-mode=symbol] {\newton\metre}   &Torque
  \\ 
  $\bf{v_\text{i}}$ & \si[per-mode=symbol] {\metre\per\second}   &Initial velocity
  \\ 
  $\bf{v_\text{f}}$ & \si[per-mode=symbol] {\metre\per\second}   &Final velocity
  \\ 
  $V$ & \si[per-mode=symbol] {\metre^{3}}   &Volume
  \\ 
  $X$ & \si[per-mode=symbol] {\metre}   &Displacement of spring from equilibrium
  \\ 
  $\zeta$ & unitless  &Damping coefficient
  \\ 
  \bottomrule
\end{longtable}

%1.3 Abbreviations and Acronyms
\subsection{Abbreviations and Acronyms}

\renewcommand{\arraystretch}{1.2}
\begin{tabular}{l l} 
  \toprule		
  \textbf{symbol} & \textbf{description}\\
  \midrule 
  A & Assumption\\
  COM & Center of mass\\
  DD & Data Definition\\
  GD & General Definition\\
  GS & Goal Statement\\
  IM & Instance Model\\
  LC & Likely Change\\
  ODE & Ordinary Differential Equation\\
  R & Requirement\\
  SRS & Software Requirements Specification\\
  T & Theoretical Model\\
  \bottomrule
\end{tabular}\\


%%%%%%%%%%%%%%%%%%%%%%%%
%
%	2.) Introduction 
%
%%%%%%%%%%%%%%%%%%%%%%%%

\section{Introduction}
Due to the rising cost of developing video games, developers are looking for
ways to save time and money for their projects. Using an open source physics
library that is reliable and free, will cut down development costs and lead 
to better quality products.

The following section provides an overview of the Software Requirements
Specification (SRS) for Chipmunk2D, an open source 2D rigid body physics library. 
This section explains the purpose of this document,
the scope of the system, and the organization of the document.

%2.1 Purpose of Document
\subsection{Purpose of Document}

This document describes the modeling of an open source
2D rigid body physics library used for games. The goals and the theoretical
models used in Chipmunk2D are provided. This document is intended to be 
used as a reference to provide all necessary information to understand and verify
the model.

This document will be used as a starting point for subsequent development
phases, including writing the design specification and the software verification
and validation plan. The design document will show how the requirements are 
to be realized. The verification and validation plan will show the steps that will 
be used to increase confidence in the software documentation and the 
implementation.

%2.2 Scope of Requirements
\subsection{Scope of Requirements} 

The scope of the requirements includes the physical simulation of 2D rigid bodies 
acted on by forces. Given 2D rigid bodies, Chipmunk2D is intended to 
simulate how these rigid bodies interact with one another .

%2.3 Organization of Document
\subsection{Organization of Document}
The organization of this document follows the template for an SRS for scientific
computing software proposed by~\cite{Koothoor2013} and \cite{SmithAndLai2005}.
The presentation follows the standard pattern of presenting goals, theories,
definitions, and assumptions.  For readers that would like a more bottom up
approach, they can start reading the instance models in Section
\ref{sec_instance} and trace back to find any additional information they
require. 

The goal statements are refined to the theoretical models, and theoretical
models to the instance models.  


%%%%%%%%%%%%%%%%%%%%%%%%
%
%	3.) General System Description
%
%%%%%%%%%%%%%%%%%%%%%%%%

\section{General System Description}

This section provides general information about the system,
identifies the interfaces between the system and its environment, and describes
the user characteristics and the system constraints.

%3.1 User Characteristics
\subsection{User Characteristics}

The end user of Chipmunk2D should have an understanding 
of first year programming concepts and an understanding of high school
physics.

%3.2 System Constraints
\subsection{System Constraints}

There are no system constraints.


%%%%%%%%%%%%%%%%%%%%%%%%
%
%	4.) Specific System Description
%
%%%%%%%%%%%%%%%%%%%%%%%%

\section{Specific System Description}

This section first presents the problem description, which gives a high-level
view of the problem to be solved. This is followed by the solution
characteristics specification, which presents the assumptions, theories, 
and definitions that are used for the physics library.

%4.1 Problem Description
\subsection{Problem Description} \label{Sec_pd}

Creating a gaming physics library is a difficult task. Games need 
physics libraries that simulate objects acting under physics conditions,
while at the same time they need to be efficient and fast enough to
work in soft real-time during the game. Developing a physics library from scratch
takes a long period of time and is very costly. These barriers of entry make it
difficult for game developers to include physics in their products. There are
a few free, open source and high quality physics libraries available to be used
for consumer products (section \ref{sec_otss}). By creating a simple, lightweight, fast, and portable 2D
rigid body physics library, game development will be more accessible to the
masses and higher quality products will be produced.

% 4.1.1 Terminology and Definitions
\subsubsection{Terminology and  Definitions}

This subsection provides a list of terms that are used in the subsequent
sections and their meaning, with the purpose of reducing ambiguity and making it
easier to correctly understand the requirements:

\begin{itemize}

\item Rigid Body: A solid body in which deformation is neglected.
\item Elasticity: Ratio of the velocities of the two colliding objects after and
before the collision.
\item Center of Mass: The mean location of the distribution of mass of the object
\item  Cartesian coordinates: a coordinate system that specifies each point uniquely in
 a plane by a pair of numerical coordinates
\item  Right-handed coordinate system: A coordinate system where the positive z-axis comes out of the screen.

\end{itemize}

% 4.1.2 Physical System Description
\subsubsection{Physical System Description}

N/A

% 4.1.3 Goal Statements
\subsubsection{Goal Statements}

\begin{itemize}

\item[GS\refstepcounter{goalnum}\thegoalnum \label{G_gravity}:] Given the initial position and
velocity, determine the new positions and velocities over a period of time of a
2D rigid body acted upon by a force


\item[GS\refstepcounter{goalnum}\thegoalnum  \label{G_detectCollision}:] Given the initial positions and
velocities, determine if any of the rigid bodies will collide with one another over a period of time

\item[GS\refstepcounter{goalnum}\thegoalnum  \label{G_collision}:] Given the initial positions and
velocities, determine the new positions and velocities over a period of time of
2D rigid bodies that have undergone a collision

\item[GS\refstepcounter{goalnum}\thegoalnum \label{G_rotation}:] Given the initial orientation and
angular velocity, determine the new orientation and angular velocity over a
period of time of a 2D rigid body acted upon by a force

\item[GS\refstepcounter{goalnum}\thegoalnum \label{G_constraint}:] Given the initial positions and
velocities, determine the positions and velocities over a period of time of 2D
rigid bodies with constraints or joints between them

\end{itemize}

%4.2 Solution Characteristics Specification
\subsection{Solution Characteristics Specification}

% 4.2.1 Assumptions
\subsubsection{Assumptions}
This section simplifies the original problem and helps in developing the
theoretical model by filling in the missing information for the physical
system. The numbers given in the square brackets refer to the data definition,
or the instance model, in which the respective assumption is used.

\begin{itemize}
\item [A\refstepcounter{assumpnum}\theassumpnum \label{A_rigid}:] All objects are rigid bodies

\item [A\refstepcounter{assumpnum}\theassumpnum \label{A_2d}:] All objects are 2D
	
\item [A\refstepcounter{assumpnum}\theassumpnum \label{A_damping}:] The damping coefficient is constant throughout the simulation 

\item[A\refstepcounter{assumpnum}\theassumpnum \label{A_cartesian}:]  The library uses a Cartesian coordinate system 

\item [A\refstepcounter{assumpnum}\theassumpnum \label{A_right}:] The axes are defined using a right hand system
\end{itemize}

% 4.2.2 Theoretical Models
\subsubsection{Theoretical Models}\label{sec_theoretical}

This section focuses on the general equations and laws that the physics
library is based on.

~\newline

%Theoretical Model 1 - T1 - Netwon's Second Law of Motion
\noindent
\begin{minipage}{\textwidth}
\renewcommand*{\arraystretch}{1.5}
\begin{tabular}{| p{\colAwidth} | p{\colBwidth}|}
  \hline
  \rowcolor[gray]{0.9}
  Number& T\refstepcounter{theorynum}\thetheorynum \label{T_NSL}\\
  \hline
  Label&\bf Newton's second law of motion\\
  \hline
  Equation& $\bf{F} =$m$ \bf{a}$\\
  \hline
  Description &
 $\bf{F}$ = Net force applied \\
& $m$ =Mass of the object  \\
&$\bf{a}$ = Acceleration of the object \\
& The net force on an object is proportional to the objects acceleration\\
  \hline
  Source \\
  \hline
  Ref.\ By & IM\ref{IM_G},  GD\ref{GD_I}\\
  \hline
\end{tabular}
\end{minipage}\\

~\newline

%Theoretical Model 2 - T2 - Netwon's Third Law of Motion
\noindent
\begin{minipage}{\textwidth}
\renewcommand*{\arraystretch}{1.5}
\begin{tabular}{| p{\colAwidth} | p{\colBwidth}|}
  \hline
  \rowcolor[gray]{0.9}
  Number& T\refstepcounter{theorynum}\thetheorynum \label{T_NTL}\\
  \hline
  Label&\bf Newton's third law of motion\\
  \hline
  Equation&\bf  $\bf{F_\text{1}} = -\bf{F_\text{2}}$\\
  \hline
  Description &
 $\bf{F_\text{1}} $ =The force being exerted on the second object by the first  \\
& $\bf{F_\text{2}}$ =The force being exerted on the first object by the second \\
& Every action has an equal and opposite reaction\\
  \hline
  Source \\
  \hline
  Ref.\ By &GD\ref{GD_COM} \\
  \hline
\end{tabular}
\end{minipage}\\

~\newline

%Theoretical Model 3 - T3 - Netwon's Law of Universal Gravitation
\noindent
\begin{minipage}{\textwidth}
\renewcommand*{\arraystretch}{1.5}
\begin{tabular}{| p{\colAwidth} | p{\colBwidth}|}
  \hline
  \rowcolor[gray]{0.9}
  Number& T\refstepcounter{theorynum}\thetheorynum \label{T_NUG}\\
  \hline
  Label&\bf Newton's Law of universal gravitation\\
  \hline
  Equation&  $\bf{F}$ =$G$ $\frac{m_\text{1}m_\text{2}}{||\textbf{r}||^{2}}\frac{\textbf{r}}{||\textbf{r}||}$\\
  \hline
  Description & 
 $\bf{F}$ = Force between the objects\\
&$m_\text{1}$ =Mass of the first object  \\
&$m_\text{2}$ =Mass of the second object  \\
&$G$ = Gravitational constant  ($6.673*10^{-11} m^{3}kg^{-1}s^{-2}$)\\
&$||\textbf{r}||$ =  The distance between the centers of the objects\\
&$\textbf{r}$ =  The displacement vector between the centers of the objects\\
& Any two bodies in the universe attract each other with a force that is
directly proportional to the product of their masses and inversely
proportional to the square of the distance between them.\\
  \hline
  Source \\
  \hline
  Ref.\ By & IM\ref{IM_G}\\
  \hline
\end{tabular}
\end{minipage}\\

~\newline

%Theoretical Model 4 - T4 - Hooke's Law
\noindent
\begin{minipage}{\textwidth}
\renewcommand*{\arraystretch}{1.5}
\begin{tabular}{| p{\colAwidth} | p{\colBwidth}|}
  \hline
  \rowcolor[gray]{0.9}
  Number& T\refstepcounter{theorynum}\thetheorynum \label{T_HL}\\
  \hline
  Label&\bf Hooke's law\\
  \hline
  Equation&  $\bf{ F} =\textnormal{k}X$\\  
  \hline
  Description & 
 $\bf F$ = Force between the objects\\
&$\bf X$ =The distance the spring is extended or compressed from its equilibrium
position \\
&$k$ =Spring constant  \\
  \hline
  Source \\
  \hline
  Ref.\ By & \\
  \hline
\end{tabular}
\end{minipage}\\

~\newline

%Theoretical Model 5 - T5 - Torque
\noindent
\begin{minipage}{\textwidth}
\renewcommand*{\arraystretch}{1.5}
\begin{tabular}{| p{\colAwidth} | p{\colBwidth}|}
  \hline
  \rowcolor[gray]{0.9}
  Number& T\refstepcounter{theorynum}\thetheorynum \label{T_RM}\\
  \hline
  Label&\bf Equation of Rotational motion\\
  \hline
  Equation& $\boldsymbol{\tau} = \bf{I}\boldsymbol{\alpha}$\\  
  \hline
  Description &  
 $\boldsymbol{\tau}$ =Torque\\
&$\bf{I}$ =The moment of inertia\\
&$\boldsymbol{\alpha}$ =Angular acceleration  \\
  \hline
  Source \\
  \hline
  Ref.\ By &GD\ref{GD_T},GD\ref{GD_MI}\\
  \hline
\end{tabular}
\end{minipage}\\
~\newline

% 4.2.3 General Definitions
\subsubsection{General Definitions}\label{sec_gendef}

This section collects  the laws and equations that will be used in deriving the
data definitions, which in turn are used to  build the instance models.

~\newline

%General Definition 1 -GD1 - Impulse
\noindent
\begin{minipage}{\textwidth}
\renewcommand*{\arraystretch}{1.5}
\begin{tabular}{| p{\colAwidth} | p{\colBwidth}|}
  \hline
  \rowcolor[gray]{0.9}
  Number& GD\refstepcounter{defnum}\thedefnum \label{GD_I}\\
  \hline
  Label&\bf Impulse\\
  \hline
  Equation& $\bf{J}$ = $\int$ $\bf{F}$ $dt$ = $\Delta$ $\bf{P}$ = $m$ $\Delta$ $\bf{v}$\\
  \hline
  Description &  
$\bf{J}$ =Impulse\\
&$\bf{F}$ = Force \\
&$\Delta \bf{P}$ = Change in momentum \\
&$m$ =Mass of the object  \\
&$\Delta \bf{v}$ =Change in velocity of the object\\
&An impulse occurs when a force $\bf{F}$ acts over an interval of time \\
  \hline
  Source\\
  \hline
  Ref.\ By &T\ref{T_NSL}, GD\ref{GD_COM}\\
  \hline
\end{tabular}
\end{minipage}\\


%General Definition 1 - GD1 - Derivation of Impulse
\subsubsection*{Derivation of Impulse}

Newton's second law of motion states:
\begin{equation*}
\bf{F}= \textnormal{m}a = \textnormal{m} \frac{\textnormal dv}{\textnormal{dt}}
\end{equation*}

\noindent
Rearranging: 
\begin{equation*}
\int_{t_\text{1}}^{t_\text{2}} \bf{F}\textnormal{dt}= \textnormal{m}\int_{\textnormal v_\text{1}}^{\textnormal v_\text{2}}\textnormal dv
\end{equation*}

\noindent
Integrating the right hand side: 
\begin{equation*}
\int_{t_\text{1}}^{t_\text{2}} \bf{F}\textnormal{dt}= \textnormal{m}v_\text{2} - \textnormal{m}v_\text{1} = \textnormal {$m\Delta$}v
\end{equation*}

%General Definition 2 -GD2 - Conservation of momentum
\noindent
\begin{minipage}{\textwidth}
\renewcommand*{\arraystretch}{1.5}
\begin{tabular}{| p{\colAwidth} | p{\colBwidth}|}
  \hline
  \rowcolor[gray]{0.9}
  Number& GD\refstepcounter{defnum}\thedefnum \label{GD_COM}\\
  \hline
  Label&\bf Conservation of momentum\\
  \hline
  Equation&  $\sum_{k=0}^{n}$  $m_\text{k}$ $\bf{v}_\text{ik}$   = $\sum_{k=0}^{n} m_\text{k}\bf{v}_\text{fk} $\\
  \hline
  Description &  
$m_\text{k}$ =Mass of the object\\
&$\bf{v}_\text{ik}$ =Initial velocity of the object  \\
&$\bf{v}_\text{fk}$ =Final velocity of the  object\\
&In an isolated system, where the sum of external impulses acting on the system
is zero the total momentum of the objects is constant \\
  \hline
  Source\\
  \hline
  Ref.\ By &T\ref{T_NTL}, GD\ref{GD_I}, IM\ref{IM_C}\\
  \hline
\end{tabular}
\end{minipage}\\



%General Definition 2 - GD2 - Derivation
\subsubsection*{Derivation of the conservation of momentum}

When objects collide they exert an equal force on each other in opposite
directions. This is Newton's third law:
\begin{equation*}
\bf{F}_\text{1} \textnormal = -\bf{F}_\text{2}
\end{equation*}

\noindent
The objects collide with each other for the exact same amount of time (t):
\begin{equation}
\bf{F}_\text{1}\textnormal{t} = -\bf{F}_\text{2}\textnormal{t}  \label{conservation1}
\end{equation}

\noindent
The above equation is equal to the impulse:
\begin{equation*}
\bf{ F}_\text{1} = \textnormal{$\int$}{ F}_\text{1} \textnormal{dt} =  \bf{J}
\end{equation*}

\noindent
The impulse is equal to the change in momentum: 
\begin{equation}
\bf{J}= \textnormal{$\Delta$} P = \textnormal {m} \textnormal{$\Delta$} \bf{ v} \label{conservation2}
\end{equation}

\noindent
substitution \ref{conservation2} into \ref{conservation1} 
\begin{equation*}
m_\text{1}\Delta \bf v_\text{1} = - \textnormal{m}_\text{2} \textnormal{$\Delta$} v_\text{2} 
\end{equation*}

\noindent
Expanding and rearranging the above formula gives:
\begin{equation*}
m_\text{1}\bf{v}_\text{i1} + \textnormal{m}_\text{2}\bf{v}_\text{i2}=\textnormal{m}_\text{1}\bf{v}_\text{f1} + \textnormal{m}_\text{2}\bf{v}_\text{f2}
\end{equation*}

\noindent
Generalizing for multiple colliding objects: 
\begin{equation*}
\sum_{k=0}^{n} m_\text{k}\bf{v}_\text{ik}   = \textnormal{$\sum_{k=0}^{n}$} \textnormal{m}_\text{k}v_\text{fk}
\end{equation*}

~\newline 

%General Definition 3 -GD3 - Coefficient of restitution
\noindent
\begin{minipage}{\textwidth}
\renewcommand*{\arraystretch}{1.5}
\begin{tabular}{| p{\colAwidth} | p{\colBwidth}|}
  \hline
  \rowcolor[gray]{0.9}
  Number& GD\refstepcounter{defnum}\thedefnum \label{GD_COR}\\
  \hline
  Label&\bf Coefficient of restitution \\
  \hline
  Equation& $C_\text{R} = \bf \frac{v_\text{f2} - v_\text{f1}}{v_\text{i1} - v_\text{i2}}$\\
  \hline
  Description &  
$C_\text{R}$ = Coefficient of restitution \\ 
&$\bf{v_\text{i1}}$ =Initial velocity of the first object  \\
&$\bf v_\text{i2}$ = Initial velocity of the second object\\
&$\bf v_\text{f1}$ =Final velocity of the first object\\
&$\bf v_\text{f2}$ =Final velocity of the second object\\
&The coefficient of restitution determines the elasticity of the collision. 
A $C_\text{R} = 1$ results in an elastic collision,
while a $C_\text{R} < 1$ results in an inelastic collision \\
  \hline
  Source\\
  \hline
  Ref.\ By & IM\ref{IM_C}\\
  \hline
\end{tabular}
\end{minipage}\\

~\newline

%General Definition 4 -GD4 - Torque 
\noindent
\begin{minipage}{\textwidth}
\renewcommand*{\arraystretch}{1.5}
\begin{tabular}{| p{\colAwidth} | p{\colBwidth}|}
  \hline
  \rowcolor[gray]{0.9}
  Number& GD\refstepcounter{defnum}\thedefnum \label{GD_T}\\
  \hline
  Label&\bf Calculating torque from a force \\
  \hline
  Equation& $\bf \boldsymbol{\tau}=r \textnormal{$\times$} F = rF_\perp $\\ 
  \hline
  Description &  
$\boldsymbol{\tau}$=Rotational force on the object\\ 
&$\bf F$ = The force on the lever arm \\
&$\bf F_\perp$ = The force perpendicular to the lever arm \\
&$\bf r$ =The distance between the force and the axis of rotation (position vector) \\
  \hline
  Source\\
  \hline
  Ref.\ By & T\ref{T_RM}, IM\ref{IM_R}\\
  \hline
\end{tabular}
\end{minipage}\\

~\newline

%General Definition 5 -GD5 - Moment of Inertia 
\noindent
\begin{minipage}{\textwidth}
\renewcommand*{\arraystretch}{1.5}
\begin{tabular}{| p{\colAwidth} | p{\colBwidth}|}
  \hline
  \rowcolor[gray]{0.9}
  Number& GD\refstepcounter{defnum}\thedefnum \label{GD_MI}\\
  \hline
  Label&\bf  Moment of Inertia \\
  \hline
  Equation& $\bf I = \textnormal{$\sum_{i=0}^{n}m_\text{i}d^{2}_\text{pi}$}$\\
  \hline
  Description &  
$\bf I$=Moment of Inertia\\ 
&$n$ = The number of particles \\
&$m_\text{i}$ =The mass of the particle i \\
&$d_\text{pi}$ = Distance between particle i and the axis of rotation\\
  \hline
  Source\\
  \hline
  Ref.\ By& T\ref{T_RM}\\
  \hline
\end{tabular}
\end{minipage}\\

%4.2.4 Data Definitions
\subsubsection{Data Definitions}\label{sec_datadef}

This section collects and defines all the data needed to build the instance
models. The dimension of each quantity is also given.

%Data Definition 1 -DD1 - Kinematics
\noindent
\begin{minipage}{\textwidth}
\renewcommand*{\arraystretch}{1.5}
\begin{tabular}{| p{\colAwidth} | p{\colBwidth}|}
\hline
\rowcolor[gray]{0.9}
Number& DD\refstepcounter{datadefnum}\thedatadefnum \label{DD_K}\\
\hline
Label& \bf Kinematics\\
\hline
Symbol &\bf d, v\\
\hline
SI Units & $\si{\metre}$, $\si[per-mode=symbol]{\metre\per\second}$\\
\hline
Equation&$\bf a(\textnormal t) = \frac{\textnormal dv(\textnormal t)}{\textnormal{dt}}$,  $\bf v(t) = \frac{\textnormal dx(\textnormal t)}{\textnormal{dt}}$\\

\hline
Description & 
$\bf a(\textnormal t) $ = Object's acceleration over time\\
&$ \bf v(\textnormal t) $ = Object's velocity over time\\
&$\bf x(\textnormal t)$ =Object's displacement over time \\
&$t$ = Time \\
\hline
Sources& \\
\hline
Ref.\ By & T\ref{T_NSL}, IM\ref{IM_G} \\
\hline
\end{tabular}
\end{minipage}\\

%Data Definition 2 -DD2 - Kinematics with friction
\noindent
\begin{minipage}{\textwidth}
\renewcommand*{\arraystretch}{1.5}
\begin{tabular}{| p{\colAwidth} | p{\colBwidth}|}
\hline
\rowcolor[gray]{0.9}
Number& DD\refstepcounter{datadefnum}\thedatadefnum \label{DD_KD}\\
\hline
Label& \bf Kinematics with damping\\
\hline
Symbol &d, v\\
\hline
SI Units & $\si{\metre}$, $\si[per-mode=symbol]{\metre\per\second}$\\
\hline
Equation&$\bf p(\textnormal t) = p_\text{i} + v_\text{i}\textnormal{$t\zeta^\text{t}$} + \textnormal{$\frac{1}{2}$}a\textnormal{$ t^{2}$}$,
 $\bf v(\textnormal t)= v_\text{i}\textnormal{$\zeta^\text{t}$} + \bf a\textnormal t$\\
\hline
Description & 
$\Delta \bf d$ =Change in displacement \\
&$\bf p(\textnormal t)$ = Object's position over time\\
&$\bf p_\text{i}$ = Object's initial position\\
&$\bf v(\textnormal t) $ = Object's velocity over time\\
&$\Delta \bf v$ =Change in velocity \\
&$\bf v_\text{i} $ = Object's initial velocity\\
&$\bf a $ = Object's acceleration\\
&$t$ = Time \\
&$\zeta$ = The damping coefficient \\
\hline
Sources& \\
\hline
Ref.\ By & T\ref{T_NSL} \\
\hline
\end{tabular}
\end{minipage}\\

~\newline

%Data Definition 3 -DD2 -  Rotational Kinematics
\noindent
\begin{minipage}{\textwidth}
\renewcommand*{\arraystretch}{1.5}
\begin{tabular}{| p{\colAwidth} | p{\colBwidth}|}
\hline
\rowcolor[gray]{0.9}
Number& DD\refstepcounter{datadefnum}\thedatadefnum \label{DD_RK}\\
\hline
Label& \bf Rotational Kinematics \\
\hline
Symbol &$ \boldsymbol{\phi}, \boldsymbol{\omega}$\\
\hline
SI Units &$\si[per-mode=symbol]{\radian}, \si[per-mode=symbol]{\radian\per\second}$\\
\hline
Equation&$\boldsymbol{\alpha}(t) = \frac{d \boldsymbol{\omega}(t)}{dt}$, $\boldsymbol{\omega}(t) = \frac{d \boldsymbol{\phi}(t)}{dt} $\\
\hline
Description & 
$ \boldsymbol{\alpha}(t) $ = Object's angular acceleration over time\\
&$ \boldsymbol{\omega}(t) $ = Object's angular velocity over time\\
&$\boldsymbol{\phi}(t)$ =Object's orientation over time\\
&$t$ = Time\\
\hline
Sources& \\
\hline
Ref.\ By & T\ref{T_RM} \\
\hline
\end{tabular}
\end{minipage}\\


%Data Definition 3 -DD2 -  Rotational Kinematics with damping
\noindent
\begin{minipage}{\textwidth}
\renewcommand*{\arraystretch}{1.5}
\begin{tabular}{| p{\colAwidth} | p{\colBwidth}|}
\hline
\rowcolor[gray]{0.9}
Number& DD\refstepcounter{datadefnum}\thedatadefnum \label{DD_RKD}\\
\hline
Label& \bf Rotational Kinematics with damping \\
\hline
Symbol &$\boldsymbol \phi,\boldsymbol \omega$\\
\hline
SI Units &$\si[per-mode=symbol]{\radian}, \si[per-mode=symbol]{\radian\per\second}$\\
\hline
Equation&$\boldsymbol \phi(t)  = \boldsymbol \phi_\text{i} +  \boldsymbol  \omega_\text{i}t\zeta^\text{t} + \frac{1}{2}\boldsymbol \alpha t^{2}$,
$\boldsymbol \omega(t) =  \boldsymbol \omega_\text{i}\zeta^\text{t} + \boldsymbol \alpha t$\\
\hline
Description & 
$\boldsymbol \phi(t) $ = Object's  orientation over time \\
&$\boldsymbol  \phi_\text{i} $ = Object's initial  orientation\\
&$\boldsymbol  \omega(t) $ = Object's angular velocity over time\\
&$\boldsymbol  \omega_\text{i} $ = Object's initial angular velocity\\
&$\boldsymbol \alpha $ = Object's angular acceleration\\
&$t$ = Time\\
&$\zeta$ = The damping coefficient \\
\hline
Sources& \\
\hline
Ref.\ By & T\ref{T_RM} \\
\hline
\end{tabular}
\end{minipage}\\

%4.2.5 Instance Models
\subsubsection{Instance Models} \label{sec_instance}    

This section transforms the problem defined in the Section~\ref{Sec_pd} into 
one which is expressed in mathematical terms. It uses concrete symbols defined 
in Section~\ref{sec_datadef} to replace the abstract symbols in the models 
identified in the Sections~\ref{sec_theoretical} and ~\ref{sec_gendef}.

~\newline

%Instance Model 1 - IM1 - Gravity - GS1 
\noindent
\begin{minipage}{\textwidth}
\renewcommand*{\arraystretch}{1.5}
\begin{tabular}{| p{\colAwidth} | p{\colBwidth}|}
  \hline
  \rowcolor[gray]{0.9}
  Number& IM\refstepcounter{instnum}\theinstnum \label{IM_G}\\
  \hline
  Label& \bf Gravity on a 2D rigid body without damping\\
  \hline
  Input&$g$, $\bf p_\text{i}$, $\bf v_\text{i}$\\
  \hline
  Output& $\bf p\textnormal{(t)}$, $\bf v\textnormal{(t)}$ such that  \bf a$ = \frac{d\bf v \textnormal{(t)}}{dt} = 
  \begin{bmatrix}
         0 \\
       -g \\
        \end{bmatrix} $ \\
  \hline
 Description &  
$\bf a$ =Acceleration \\
&$\bf p_\text{i}$ = Initial position of the object\\
&$\bf v_\text{i}$ =Initial velocity of the object\\
&$\bf p\textnormal{(t)}$ =Position of the object as a function of time\\
&$\bf v\textnormal{(t)}$ =Velocity of the object as a function of time\\
&$t$ = Time \\
&$g$ = Gravity\\
  \hline
  Sources \\
  \hline
Ref.\ By & T\ref{T_NSL}, T\ref{T_NUG}, DD\ref{DD_K} \\
  \hline
\end{tabular}
\end{minipage}\\

\subsubsection*{Derivation of gravity}

Netwon's first law of motion states:

\begin{equation}
\bf F= \textnormal ma \label{eq:NSL}
\end{equation}

\noindent
Newton's law of Universal Gravitation states: 
\begin{equation}
\bf{F} =\textnormal{$G \frac{m_\text{1}m_\text{2}}{||\textbf{r}||^{2}}$}\frac{r}{|| r||} \label{eq:NUG}
\end{equation}

\noindent
Substituting \ref{eq:NUG} into \ref{eq:NSL}: 
\begin{equation*}
m\bf a =\textnormal{$G \frac{m_\text{1}m_\text{2}}{||\textbf{r}||^{2}}$}
\end{equation*}

\noindent
Simplifying: 
\begin{equation*}
\bf a =\textnormal{$G \frac{m_\text{1}}{||\textbf{r}||^{2}}$}
\end{equation*}


\noindent
Let \textbf{g} = $G \frac{m_\text{1}}{||\textbf{r}||^{2}}$, The acceleration due to gravity is: 
\begin{equation*}
\textbf{a}= \textbf{g}
\end{equation*}


%Instance Model 1 - IM1 - Force - GS1 
\noindent
\begin{minipage}{\textwidth}
\renewcommand*{\arraystretch}{1.5}
\begin{tabular}{| p{\colAwidth} | p{\colBwidth}|}
  \hline
  \rowcolor[gray]{0.9}
  Number& IM\refstepcounter{instnum}\theinstnum \label{IM_G}\\
  \hline
  Label& \bf Force on a 2D rigid body without damping\\
  \hline
  Input&$g$, $\bf p_\text{i}$, $\bf v_\text{i}$, $\textbf{F}$\\
  \hline
  Output& $\bf p\textnormal{(t)}$, $\bf v\textnormal{(t)}$ such that  \bf a$ = \frac{d\bf v \textnormal{(t)}}{dt} = 
  \begin{bmatrix}
         F_\text{x} \\
         F_\text{y}\\
        \end{bmatrix} $ \\
  \hline
 Description &  
$\bf a$ =Acceleration \\
&$\bf p_\text{i}$ = Initial position of the object\\
&$\bf v_\text{i}$ =Initial velocity of the object\\
&$\bf p\textnormal{(t)}$ =Position of the object as a function of time\\
&$\bf v\textnormal{(t)}$ =Velocity of the object as a function of time\\
&$t$ = Time \\
&$\textbf{F}$ = Force acting on the object\\
  \hline
  Sources \\
  \hline
Ref.\ By & T\ref{T_NSL}, T\ref{T_NUG}, DD\ref{DD_K} \\
  \hline
\end{tabular}
\end{minipage}\\

%Instance Model 2 - IM2 - Collision - GS2
\noindent
\begin{minipage}{\textwidth}
\renewcommand*{\arraystretch}{1.5}
\begin{tabular}{| p{\colAwidth} | p{\colBwidth}|}
  \hline
  \rowcolor[gray]{0.9}
  Number& IM\refstepcounter{instnum}\theinstnum \label{IM_C}\\
  \hline
  Label& \bf Collisions on 2D rigid bodies\\
  \hline
  Input&$m_\text{1}$, $m_\text{2}$, $\bf v_\text{i1}$, $\bf v_\text{i2}$, $\bf p_\text{i1}$, $\bf p_\text{i2}$. $C_\text{R}$\\
  \hline
  Output&$\bf v_\text{1}(\textnormal t)$, $\bf v_\text{2}(\textnormal t)$, $\bf p_\text{1}(\textnormal t)$, $\bf p_\text{2}(\textnormal t)$ such that:
    $\sum_{k=0}^{n}$ $m_\text{k}$$\bf{v}_\text{ik}$   = $\sum_{k=0}^{n} m_\text{k}\bf{v}_\text{fk} $,
    $C_\text{R} = \bf \frac{v_\text{f2} - v_\text{f1}}{v_\text{i1} - v_\text{i2}}$,
    $ \bf a_\text{1} = \frac{\textnormal d\bf v_\text{1}\textnormal{(t)}}{\textnormal{dt}} $,
    $ \bf a_\text{2} = \frac{\textnormal d\bf v_\text{2}\textnormal{(t)}}{\textnormal{dt}} $\\ 
  \hline
  Description &  
$\bf p_\text{i1}$ = Initial position of the first object\\
&$\bf p_\text{i2}$ = Initial position of the second object\\
&$\bf v_\text{\textnormal{i1}}$ =Initial velocity of the first object\\
&$\bf v_\text{\textnormal{i2}}$ =Initial velocity of the second object\\
&$\bf p_\text{\textnormal 1}(\textnormal t)$ =Final position of the first object as a function of time\\
&$\bf p_\text{\textnormal 2}(\textnormal t)$ =Final position of the second object as a function of time\\
&$\bf v_\text{\textnormal 1}(\textnormal t)$ =Final velocity of the first object as a function of time\\
&$\bf v_\text{\textnormal 2}(\textnormal t)$ =Final velocity of the second object as a function of time\\
&$\bf a_\text{\textnormal{1}}$ = Acceleration of the first object\\
&$\bf a_\text{\textnormal{2}}$ =Acceleration of the second object\\
&$C_\text{R}$ = Coefficient of restitution\\
  \hline
  Sources \\
  \hline
Ref.\ By & GD\ref{GD_COM}, GD\ref{GD_COR}, DD\ref{DD_K} \\
  \hline
\end{tabular}
\end{minipage}\\


%Instance Model  3 -IM3 - Rotation - GS3
\noindent
\begin{minipage}{\textwidth}
\renewcommand*{\arraystretch}{1.5}
\begin{tabular}{| p{\colAwidth} | p{\colBwidth}|}
  \hline
  \rowcolor[gray]{0.9}
  Number& IM\refstepcounter{instnum}\theinstnum \label{IM_R}\\
  \hline
  Label& \bf Rotation on a 2D rigid body\\
  \hline
  Input&$\boldsymbol \phi_\text{i}$, $\boldsymbol \omega_\text{i}$, $\bf F$, $\bf p_\text{com}$, $\bf p_\text{force}$, \textbf I\\
  \hline
  Output& $\boldsymbol \phi(t)$, $\boldsymbol \omega(t)$ such that: $\boldsymbol{\alpha}(t) = \frac{d \boldsymbol{\omega}(t)}{dt}$ \\
  \hline
  Description &  
$\boldsymbol \phi_\text{i}$ = Initial orientation of the object\\
&$\boldsymbol \omega_\text{i}$ =Initial angular velocity of the object\\
&$\bf F$ =Force acting on the object\\
&$\bf I$ =Moment of Intertia of the object\\
&$\bf p_\text{\textnormal com}$ =Center of mass of the object\\
&$\bf p_\text{\textnormal force}$ = Point the force is applied on the object\\
&$\boldsymbol \phi(\textnormal t)$ =Orientation of the object as a function of time\\
&$\boldsymbol \omega(\textnormal t)$ =Angular velocity of the object as a function of time\\
&$\boldsymbol \alpha(\textnormal t)$ =Angular acceleration of the object as a function of time\\
  \hline
  Sources \\
  \hline
Ref.\ By & T\ref{T_RM}, GD\ref{GD_T},DD\ref{DD_RK} \\
  \hline
\end{tabular}
\end{minipage}\\

\subsubsection{Detailed derivation of  Rotation on a 2D rigid body}
Using GD\ref{GD_T} to calculate the toruqe
\begin{equation*}
 \boldsymbol{\tau}=\textbf r{\times}\textbf F =\textbf r \textbf F_\perp  =( \textbf  p_\text{com}-  \textbf p_\text{ force})  \textbf F_\perp 
\end{equation*}

\noindent
The angular acceleration can be calculated (T\ref{T_RM}):
\begin{equation*}
 \boldsymbol{\alpha}= \frac{\boldsymbol \tau}{\textbf I} =  \frac{( \textbf  p_\text{com}-  \textbf p_\text{ force})  \textbf F_\perp }{\textbf I}
\end{equation*}

\noindent
Using DD\ref{DD_RK} The object's angular velocity and orientation can be calculated:
\begin{equation*}
\boldsymbol{\alpha}(t) = \frac{d \boldsymbol{\omega}(t)}{dt}
\end{equation*}

\noindent
Integrating both sides gives:
\begin{equation*}
\boldsymbol{\omega}(t) = \boldsymbol{\alpha}t +\boldsymbol \omega_\text{i}
\end{equation*}

\noindent
Using the above angular velocity to calculate the orientation:
\begin{equation*}
\boldsymbol{\omega}(t) = \frac{d \boldsymbol{\phi}(t)}{dt}
\end{equation*}

\noindent
Integrating both sides gives:
\begin{equation*}
\boldsymbol{\phi}(t) = \frac{1}{2}\boldsymbol{\alpha}t^{2} +\boldsymbol \omega_\text{i}t + \boldsymbol \phi_\text{i}
\end{equation*}
~\newline

%Instance Model 4 - IM4 - Connections - Rods - GS4
\noindent
\begin{minipage}{\textwidth}
\renewcommand*{\arraystretch}{1.5}
\begin{tabular}{| p{\colAwidth} | p{\colBwidth}|}
  \hline
  \rowcolor[gray]{0.9}
  Number& IM\refstepcounter{instnum}\theinstnum \label{IM_CRod}\\
  \hline
  Label& \bf Connections and joints on 2D rigid bodies \\
  \hline
Ref.\ By &  The instance models for constraints are located in the
  SRS Constraints document\\
  \hline
\end{tabular}
\end{minipage}\\

%4.2.6 Data Constraints
\subsubsection{Data Constraints} \label{sec_DataConstraints}    

Table~\ref{TblInputVar} and \ref{TblOutputVar} show the data constraints on the
input and output variables, respectively.  The column physical constraints gives 
the physical limitations on the range of values that can be taken by the
variable.  The constraints are conservative, to give the user of the model the
flexibility to experiment with unusual situations.  The column of typical values
is intended to provide a feel for a common scenario. 

%Table of input variables
\begin{table}[!h]
\caption{Input Variables} \label{TblInputVar}
\renewcommand{\arraystretch}{1.2}
\noindent \begin{longtable}{l l l c} 
  \toprule
  \textbf{Var} & \textbf{Physical Constraints} & \textbf{Typical Value} \\
  \midrule 
  $\bf a$			& None			& 9.8 \si[per-mode=symbol]	{\metre\per\second^{2}}	
  \\
  $\bf d$			& None			& 0.412 \si[per-mode=symbol] {\metre}	
  \\
  $\bf v$			 & None 			& 0.05 \si[per-mode=symbol] {\metre\per\second}	
  \\
  $m$ 			& $m \ge 0$			& 1.2 \si[per-mode=symbol] {\kilo\gram}	
  \\
  $C_\text{R}$  	& $ 0 \le C_\text{R} \le 1$	& 0.8 
  \\	
  $\boldsymbol \alpha$ 		& None 			& 1.6 \si[per-mode=symbol] {\radian\per\second^2}
  \\
  $\boldsymbol \phi$		 & $ 0 \le \phi< 2\pi $	&$\frac{\pi}{2} $ \si[per-mode=symbol] {\radian}
  \\	
  $\boldsymbol \omega$ 		& None 			& 2.1 \si[per-mode=symbol] {\radian\per\second}
  \\
  $\zeta$ 		&$ \zeta \ge 0 $			& 0.91
  \\
  $L$ 			& $L\ge0$			&44.2 \si[per-mode=symbol] {\metre}
  \\
  $k$			& $k\ge0$			&1.4 \si[per-mode=symbol] {\newton\per\metre}
  \\
  $V$ 			& $V>0$ (*)			& 22 \si[per-mode=symbol] {\metre^{3}} 
  \\
  $\rho$ 		& $\rho\ge 0$ 		&1 \si[per-mode=symbol] {\kilo\gram\per\metre^{3}} 
  \\
  \bottomrule
\end{longtable}
\end{table}

\noindent \begin{description}
\item[(*)] These quantities cannot be equal to zero, or there will be a divide by
  zero in the model.
\end{description}

%Table of output variables
\begin{table}[!h]
\caption{Output Variables} \label{TblOutputVar}
\renewcommand{\arraystretch}{1.2}
\noindent \begin{longtable}{l l} 
  \toprule
  \textbf{Var} & \textbf{Physical Constraints} \\
  \midrule 
  $\bf v$ 		& None
  \\
  $\bf p$ 		& None
  \\
  $\bf F$		 & None
  \\
  $\boldsymbol \phi$	&$ 0 \le \phi< 2\pi $
  \\
  $\boldsymbol \omega$	& None
  \\
  \bottomrule
\end{longtable}
\end{table}

%%%%%%%%%%%%%%%%%%%%%%%%
%
%	5.) Requirements 
%
%%%%%%%%%%%%%%%%%%%%%%%%

\section{Requirements}

This section provides the functional requirements, the business tasks that the
software is expected to complete, and the nonfunctional requirements, the
qualities that the software is expected to exhibit.

%5.1 Functional Requirements
\subsection{Functional Requirements}

\noindent
\begin{itemize}

\item[R\refstepcounter{reqnum}\thereqnum \label{R_Space}:] Create a space for all of the rigid bodies in the physical simulation 
to interact in.  
\item[R\refstepcounter{reqnum}\thereqnum \label{R_Rigid}:] Input the initial mass, velocities, positions, orientations, angular
velocities, and constraints of objects
\item[R\refstepcounter{reqnum}\thereqnum \label{R_Shape}:] Input the surface properties of the objects such as friction or 
elasticity
\item[R\refstepcounter{reqnum}\thereqnum \label{R_InputConstraint}:] Input the constraints of the objects 
\item[R\refstepcounter{reqnum}\thereqnum \label{R_CheckInputs}:] Verify that the inputs satisfy the required physical constraints
\item[R\refstepcounter{reqnum}\thereqnum \label{R_Force}:] Determine the position and velocities over a period of time of the 2D
rigid bodies acted upon by a force 
\item[R\refstepcounter{reqnum}\thereqnum \label{R_CheckCollision}:] Determine if any of the rigid bodies in the space have collided
\item[R\refstepcounter{reqnum}\thereqnum \label{R_Collision}:] Determine the position and velocities over a period of time of 2D rigid
bodies that have undergone a collision
\item[R\refstepcounter{reqnum}\thereqnum \label{R_Rotation}:] Determine the orientation and angular velocities over a period of time of
the 2D rigid bodies
\item[R\refstepcounter{reqnum}\thereqnum \label{R_Constraints}:] Determine the position and velocities over a period of time of 2D rigid
bodies with constraints or joints between them
\item[R\refstepcounter{reqnum}\thereqnum \label{R_Query}:] Allow the user to query the space and return information about the 
rigid bodies. 

\end{itemize} 

%5.2 Nonfunctional Requirements
\subsection{Nonfunctional Requirements}
Games are resource intensive, so performance is a high priority.
Other non-functional requirements that are a priority are: correctness,
understandability, portability, reliability, and maintainability. 


%%%%%%%%%%%%%%%%%%%%%%%%
%
%	6.) Likely Changes 
%
%%%%%%%%%%%%%%%%%%%%%%%%

\section{Likely Changes}    


%%%%%%%%%%%%%%%%%%%%%%%%
%
%	7.) Off the Shelf Solution
%
%%%%%%%%%%%%%%%%%%%%%%%%

\section{Off the Shelf Solutions}   \label{sec_otss}
As mentioned in section \ref{Sec_pd}, there already exists free open source game
physics libraries. Similar 2D physics libraries are:
\begin{itemize} 
\item Box2D   \url{http://box2d.org/}
\item Nape Physics Engine  \url{http://napephys.com/}
\end{itemize}

\noindent
Free open source 3D game physics libraries include:
\begin{itemize} 
\item Bullet   \url{http://bulletphysics.org/}
\item Open Dynamics Engine  \url{http://www.ode.org/}
\item Newton Game Dynamics  \url{http://newtondynamics.com/}
\end{itemize}

\bibliographystyle {plain}
\bibliography {../Physics_Game_Library}

\end{document}
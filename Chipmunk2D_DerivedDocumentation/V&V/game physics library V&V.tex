\documentclass[12pt]{article}

\usepackage{bm}
\usepackage{amsmath, mathtools}
\usepackage{amsfonts}
\usepackage{amssymb}
\usepackage{graphicx}
\usepackage{colortbl}
\usepackage{xr}
\usepackage{hyperref}
\usepackage{longtable}
\usepackage{xfrac}
\usepackage{tabularx}
\usepackage{float}
\usepackage{siunitx}
\usepackage{booktabs}

%\usepackage{refcheck}

\hypersetup{
    bookmarks=true,         % show bookmarks bar?
      colorlinks=true,       % false: boxed links; true: colored links
    linkcolor=red,          % color of internal links (change box color with linkbordercolor)
    citecolor=green,        % color of links to bibliography
    filecolor=magenta,      % color of file links
    urlcolor=cyan           % color of external links
}
\newcommand{\NN}[1]{{\color{red}#1}}
\newcommand{\WSS}[1]{{\color{blue}#1}}

\newcommand{\colZwidth}{1.0\textwidth}
\newcommand{\blt}{- } %used for bullets in a list
\newcommand{\colAwidth}{0.13\textwidth}
\newcommand{\colBwidth}{0.82\textwidth}
\newcommand{\colCwidth}{0.1\textwidth}
\newcommand{\colDwidth}{0.05\textwidth}
\newcommand{\colEwidth}{0.8\textwidth}
\newcommand{\colFwidth}{0.17\textwidth}
\newcommand{\colGwidth}{0.5\textwidth}
\newcommand{\colHwidth}{0.28\textwidth}
\newcounter{defnum} %Definition Number
\newcommand{\dthedefnum}{GD\thedefnum}
\newcommand{\dref}[1]{GD\ref{#1}}
\newcounter{datadefnum} %Datadefinition Number
\newcommand{\ddthedatadefnum}{DD\thedatadefnum}
\newcommand{\ddref}[1]{DD\ref{#1}}
\newcounter{theorynum} %Theory Number
\newcommand{\tthetheorynum}{T\thetheorynum}
\newcommand{\tref}[1]{T\ref{#1}}
\newcounter{tablenum} %Table Number
\newcommand{\tbthetablenum}{T\thetablenum}
\newcommand{\tbref}[1]{TB\ref{#1}}
\newcounter{assumpnum} %Assumption Number
\newcommand{\atheassumpnum}{P\theassumpnum}
\newcommand{\aref}[1]{A\ref{#1}}
\newcounter{goalnum} %Goal Number
\newcommand{\gthegoalnum}{P\thegoalnum}
\newcommand{\gsref}[1]{GS\ref{#1}}
\newcounter{instnum} %Instance Number
\newcommand{\itheinstnum}{IM\theinstnum}
\newcommand{\iref}[1]{IM\ref{#1}}
\newcounter{reqnum} %Requirement Number
\newcommand{\rthereqnum}{P\thereqnum}
\newcommand{\rref}[1]{R\ref{#1}}
\newcounter{lcnum} %Likely change number
\newcommand{\lthelcnum}{LC\thelcnum}
\newcommand{\lcref}[1]{LC\ref{#1}}

\newcommand{\tclad}{T_\text{CL}}
\newcommand{\degree}{\ensuremath{^\circ}}
\newcommand{\progname}{SWHS}


\usepackage{fullpage}

\begin{document}

\title{Verification and Validation Plan for Open Source Physics Game Library} 
\author{Alex Halliwushka}
\date{\today}
	
\maketitle

\tableofcontents

%%%%%%%%%%%%%%%%%%%%%%%%
%
%	1.) General Information 
%
%%%%%%%%%%%%%%%%%%%%%%%%

\section{General Informationl}
The following section provides an overview of the Verification and Validation (V\&V) Plan 
for an open source physics library for video games. This section explains the purpose of this
document, the scope of the system, common definitions, acronyms and abbreviations that are used
in the document, and an overview of the following sections

%1.1 Purpose
\subsection{Purpose}
The main purpose of this document is to describe the verification and validation 
process that will be used to test an open source physics game library.
This document is indented to be used as a reference for all future testing and will
be used to increase confidence in the software implementation.  

This document will be used as a starting point for the verification and validation report. The 
test cases presented within this document will be executed and the output will be analyzed to 
determine if the software is implemented correctly.  


%1.2 Scope
\subsection{Scope}


%1.3  Definitions, Acronyms, and abbreviations 
\subsection{Definitions, Acronyms, and Abbreviations }

\renewcommand{\arraystretch}{1.2}
\begin{tabular}{l l} 
  \toprule		
  \textbf{symbol} & \textbf{description}\\
  \midrule 
  QA		&Quality assurance\\
  SRS		&Software requirements specification\\
  V\&V		& Verification and validation\\
  V\&VP 	& Verification and validation plan\\
  V\&VR 	& Verification and validation report\\
  \bottomrule
\end{tabular}\\

%1.4 Overview of Document
\subsection{Overview of Document }
The following sections provide more detail about the V\&V of an open source 2D phyiscs rigid body
library. Information about the testing process is provided, and the software specifications
that were discussed in the SRS document are stated.  The evaluation process that will be followed during 
testing is outlined, and test cases for both the system testing and unit testing are provided 

%%%%%%%%%%%%%%%%%%%%%%%%
%
%	2.) Plan
%
%%%%%%%%%%%%%%%%%%%%%%%%

\section{Plan}
This section provides a description of the software that is being tested, the team that will
perform the testing, the milestones for the testing phase, and the budget allocated to the testing. 

%2.1 Software Description
\subsection{Software Description}
The software being tested is an open source 2D rigid body physics library used for games. Given the size, shape 
and location from the user, the software constructs 2D rigid bodies and simulates how the rigid bodies react to 
forces and how they interact with one another.  

%2.2 Test Team
\subsection{Test Team} 
The team that will execute the test cases, write and review the V\&VR consist of:

\begin{itemize}
 \item Alex Halliwushka. 
 \item Dr. Spencer Smith
 \item Nolan Driessen
\end{itemize}  

%2.3 Milestones
\subsection{Milestones}

%2.3.1 Location
\subsubsection{Location}
The location that the testing will be performed is Hamilton Ontario. The institution that
will be performing the testing is McMaster University. 


%2.3.1 Dates and Deadlines
\subsubsection{Dates and Deadlines}
Test Case:
~\newline
The creation of the test cases for both system testing and unit testing is scheduled to begin June $1^\text{st}$ 2015.
The deadline for the creation of the test cases is June 15th 2015. 
~\newline
~\newline
Test Case Implementation:
~\newline
Implementing code for the automation of the unit testing is scheduled to being June 15th 2015. The implementation period
is expected to last approximately two weeks and has a deadline of June 30th 2015.
~\newline
~\newline
Verification and Validation Report:
~\newline
The writting of the V\&VR is scheduled to begin July 1st 2015 and end on July 15th 2015. 

%2.4 Budget
\subsection{Budget}
The budget for the testing of this system is being funded by McMaster University and NSERC

%%%%%%%%%%%%%%%%%%%%%%%%
%
%	3.) Software Specification
%
%%%%%%%%%%%%%%%%%%%%%%%%

\section{ Software Specification}
This section provides the functional requirements, the business tasks that the
software is expected to complete, and the nonfunctional requirements, the
qualities that the software is expected to exhibit.

%3.1 Functional Requirements
\subsection{Functional Requirements}

\noindent
\begin{itemize}
\item Input the initial mass, velocities, positions, orientations, angular velocities, and constraints of objects  
\item Verify that the inputs satisfy the requried physical constraints 
\item Determine the position and velocities over a period of time of the 2D rigid bodies acted upon by gravity
\item Determine the position and velocities over a period of time of 2D rigid bodies that have undergone a collision
\item Determine the orientation and angular velocities over a period of time of the 2D rigid bodies
\item Determine the position and velocities over a period of time of 2D rigid bodies with constraints or joints between them
\end{itemize} 

%3.2 Nonfunctional Requirements
\subsection{Nonfunctional Requirements}
Games are very resource intensive, so performance is a high priority.
Other non-functional requirements that are a priority are: correctness,
understandability, portability, reliability, and maintainability. 


%%%%%%%%%%%%%%%%%%%%%%%%
%
%	4.) Evaluation
%
%%%%%%%%%%%%%%%%%%%%%%%%

\section{Evaluation}
This section first presents the methods and constraints that are to be used during
the evaluation process. This is followed by how the data obtained by the testing will be 
evaluated, which includes: how the data will be recorded, how to move from one test
to the next, and how to determine if the test was successful. 

%4.1 Methods and Constraints
\subsection{Methods and Constraints} 

%4.1.1 Methodology
\subsubsection{Methodology} 
The testing of the physics game library will be seperated into two sections: system testing and unit testing.

The system testing will be done manually, the tester will set up the initial conditions as described in the test cases
and compare the actual results of the test to the expected results. If the results match then the test passed, otherwise
the test failed.  

The unit testing will be automated. The tester will implement the unit tests into the code using a unit testing framework. Once 
the unit tests are implemented, the software will be run and any incorrect results will be outputted by the system

% 4.1.2 Extent of Testing
\subsubsection{Extent of Testing}
The extent of testing that will be employed is extensive testing. The unit test cases below provide complete code coverage and 
will increase confidence in the verification of the software. The system test cases increase confidence in the validation of the system

% 4.1.3 Test Tools
\subsubsection{Test Tools}
A unit testing framework will be used to implement the unit test cases and run them automatically.

% 4.1.4 Testing Constraints
\subsubsection{Testing Constraints}
There are currently no anticipated limitations on the testing

% 4.2  Data Evaluation
\subsection{Data Evaluation}

% 4.2.1 Data Recording
\subsubsection{Data Recording}
After each test is run the results of the test should be recorded in the following format: 
~\newline
Test ID: 
~\newline
Input:
~\newline
Expected Output:
~\newline
Actual Output:
~\newline
Result: 

% 4.2.2 Test Progression
\subsubsection{Test Progression}

For the system test cases: Follow the preperation instructions given for the test case to get the system
initialized correctly. Follow the procedures given for the test case and use the inputs provided. Run the test and
record the results, record any dsicrepancies between the actual output and the expected output. Move on to the next
test case and repeat the process again.

% 4.2.3 Testing Criteria
\subsubsection{Testing Criteria}
The actual results of each test will be evaluated against the expected results to see if the software is working as 
intended.

% 4.2.4 Test Data Reduction
\subsubsection{Testing Data Reduction}
The results of the test data will be evaluated on a PASS/FAIL basis. If the actual results match the expected
results the test will be considered a PASS, otherwise the test is considered a FAIL. 


%%%%%%%%%%%%%%%%%%%%%%%%
%
%	5.) System Test Description
%
%%%%%%%%%%%%%%%%%%%%%%%%

\section{System Test Description}

%5.1 Test identifier 
\subsection{Gravity with no initial velocity}
% 5.1.1 Means of Control
\subsubsection{Means of Control}
Manual
% 5.1.2 Input
\subsubsection{Input}
Use the input stated in \ref{testGravity1}

% 5.1.3 Expected Output
\subsubsection{Expected Output}
$v_\text{x}(t) = 0$ , $v_\text{y}(t) = -9.8t$ \\
$p_\text{x}(t) = 0$ , $p_\text{y}(t) = 50 - 4.9t^{2}$
 
% 5.1.4 Test  Procedures
\subsubsection{Procedure}
Record the position and velocity of the object as it falls. Determine if the 
position and velocity follow the equations in the expected output.  

% 5.1.5  Preperation
\subsubsection{Preperation}
Create an object and give it the initial values indicated in \ref{testGravity1}.
Output the object's position and velocity at 0.5 s intervals.


%5.2 Test identifier 
\subsection{Gravity with initial velocity in y}
% 5.2.1 Means of Control
\subsubsection{Means of Control}
Manual
% 5.2.2 Input
\subsubsection{Input}
Use the input stated in \ref{testGravity2}

% 5.2.3 Expected Output
\subsubsection{Expected Output}
$v_\text{x}(t) = 0$ , $v_\text{y}(t) =  2 -9.8t$ \\
$p_\text{x}(t) = 0$ , $p_\text{y}(t) = 50 + 2t - 4.9t^{2}$
 
% 5.2.4 Test  Procedures
\subsubsection{Procedure}
Record the position and velocity of the object as it falls. Determine if the 
position and velocity follow the equations in the expected output.  

% 5.2.5  Preperation
\subsubsection{Preperation}
Create an object and give it the initial values indicated in \ref{testGravity2}.
Output the object's position and velocity at 0.5 s intervals.


%5.3 Test identifier 
\subsection{Gravity with initial velocity in x}
% 5.3.1 Means of Control
\subsubsection{Means of Control}
Manual
% 5.3.2 Input
\subsubsection{Input}
Use the input stated in \ref{testGravity3}

% 5.3.3 Expected Output
\subsubsection{Expected Output}
$v_\text{x}(t) = 5.75$ , $v_\text{y}(t) = -9.8t$ \\
$p_\text{x}(t) = 5.75t$ , $p_\text{y}(t) = 50 - 4.9t^{2}$
 
% 5.3.4 Test  Procedures
\subsubsection{Procedure}
Record the position and velocity of the object as it falls. Determine if the 
position and velocity follow the equations in the expected output.  

% 5.3.5  Preperation
\subsubsection{Preperation}
Create an object and give it the initial values indicated in \ref{testGravity3}.
Output the object's position and velocity at 0.5 s intervals.


%5.4 Test identifier 
\subsection{Gravity with initial velocity in x and y}
% 5.4.1 Means of Control
\subsubsection{Means of Control}
Manual
% 5.4.2 Input
\subsubsection{Input}
Use the input stated in \ref{testGravity4}

% 5.4.3 Expected Output
\subsubsection{Expected Output}
$v_\text{x}(t) = 15.9$ , $v_\text{y}(t) = 4.8  -9.8t$ \\
$p_\text{x}(t) = 15.9t$ , $p_\text{y}(t) = 50 + 4.8t  -4.9t^{2}$
 
% 5.4.4 Test  Procedures
\subsubsection{Procedure}
Record the position and velocity of the object as it falls. Determine if the 
position and velocity follow the equations in the expected output.  

% 5.4.5  Preperation
\subsubsection{Preperation}
Create an object and give it the initial values indicated in \ref{testGravity4}.
Output the object's position and velocity at 0.5 s intervals.


%5.5 Test identifier 
\subsection{Parabolic motion with damping}
% 5.5.1 Means of Control
\subsubsection{Means of Control}
Manual
% 5.5.2 Input
\subsubsection{Input}
Use the input stated in \ref{testGravity5}

% 5.5.3 Expected Output
\subsubsection{Expected Output}
$v_\text{x}(t) = 6.32*(0.95)^{t}$ , $v_\text{y}(t) = -1.77*(0.95)^{t}  -9.8t$ \\
$p_\text{x}(t) = 6.32t*(0.95)^{t}$ , $p_\text{y}(t) = 50  -1.77t*(0.95)^{t}  -4.9t^{2}$
 
% 5.5.4 Test  Procedures
\subsubsection{Procedure}
Record the position and velocity of the object as it falls. Determine if the 
position and velocity follow the equations in the expected output.  

% 5.5.5  Preperation
\subsubsection{Preperation}
Create an object and give it the initial values indicated in \ref{testGravity5}.
Output the object's position and velocity at 0.5 s intervals.



%5.6 Test identifier 
\subsection{Conservation of Momentum}
% 5.6.1 Means of Control
\subsubsection{Means of Control}
Manual
% 5.6.2 Input
\subsubsection{Input}
Use the input stated in \ref{testCollision1}

% 5.6.3 Expected Output
\subsubsection{Expected Output}

$m_\text{1}\bf{v}_\text{i1} + \textnormal{m}_\text{2}\bf{v}_\text{i2}=\textnormal{m}_\text{1}\bf{v}_\text{f1} + \textnormal{m}_\text{2}\bf{v}_\text{f2}$\\
 $C_\text{R} = \bf \frac{v_\text{f2} - v_\text{f1}}{v_\text{i1} - v_\text{i2}}$\\

\noindent
rearranging and substituting the above two equations gives the velocity right after the collision:\\ 
$v_\text{f1} =  \frac{m_\text{1}v_\text{i1} + m_\text{2}v_\text{i2}  + m_\text{2}( C_\text{R}(v_\text{i2} -v_\text{i1}))}{m_\text{1} + m_\text{2}} $
$v_\text{f2} =  \frac{m_\text{1}v_\text{i1} + m_\text{2}v_\text{i2}  + m_\text{1}( C_\text{R}(v_\text{i1} -v_\text{i2}))}{m_\text{1} + m_\text{2}} $
 
% 5.6.4 Test  Procedures
\subsubsection{Procedure}
Record the velocity of the objects right after they collide. Determine if the 
velocities are equal to the values in the expected output.  

% 5.6.5  Preperation
\subsubsection{Preperation}
Create two objects and give them the initial values indicated in \ref{testCollision1}.
Output the objects' velocities right after the collision


%5.6 Test identifier 
\subsection{Perfectly Elastic Collision}
% 5.6.1 Means of Control
\subsubsection{Means of Control}
Manual
% 5.6.2 Input
\subsubsection{Input}
Use the input stated in \ref{testCollision2}

% 5.6.3 Expected Output
\subsubsection{Expected Output}

$m_\text{1}\bf{v}_\text{i1} + \textnormal{m}_\text{2}\bf{v}_\text{i2}=\textnormal{m}_\text{1}\bf{v}_\text{f1} + \textnormal{m}_\text{2}\bf{v}_\text{f2}$\\
\noindent
$\frac{1}{2}m_\text{1}v_\text{i1}^{2} + \frac{1}{2}m_\text{2}v_\text{i2}^{2} =  \frac{1}{2}m_\text{1}v_\text{f1}^{2} + \frac{1}{2}m_\text{2}v_\text{f2}^{2} $\\

% 5.6.4 Test  Procedures
\subsubsection{Procedure}
Record the velocity of the objects right after they collide. Determine if the 
velocities are equal to the values in the expected output.  

% 5.6.5  Preperation
\subsubsection{Preperation}
Create two objects and give them the initial values indicated in \ref{testCollision2}.
Output the objects' velocities right after the collision


%5.7 Test identifier 
\subsection{Rotation with no initial angular velocity}
% 5.7.1 Means of Control
\subsubsection{Means of Control}
Manual
% 5.7.2 Input
\subsubsection{Input}
Use the input stated in \ref{testRotation1}

% 5.7.3 Expected Output
\subsubsection{Expected Output}

$\omega(t) = \frac{\pi}{10}t$, 
$\phi(t) = \frac{\pi}{4} + \frac{\pi}{20}t^2$

% 5.7.4 Test  Procedures
\subsubsection{Procedure}
Record the orientation and the angular velcoity of the object as it rotates. Determine if the 
orientation and the angular velocitiy follow the equations in the expected output.  

% 5.7.5  Preperation
\subsubsection{Preperation}
Create an objects and give them the initial values indicated in \ref{testRotation1}.
Output the objects' velocities right after the collision


%5.8 Test identifier 
\subsection{Rotation with angular velocity}
% 5.8.1 Means of Control
\subsubsection{Means of Control}
Manual
% 5.8.2 Input
\subsubsection{Input}
Use the input stated in \ref{testRotation2}

% 5.8.3 Expected Output
\subsubsection{Expected Output}

$\omega(t) = \frac{\pi}{3} - \frac{\pi}{12}t$, 
$\phi(t) = \frac{\pi}{4} + \frac{\pi}{3}t + \frac{\pi}{24}t^2$

% 5.8.4 Test  Procedures
\subsubsection{Procedure}
Record the orientation and the angular velcoity of the object as it rotates. Determine if the 
orientation and the angular velocitiy follow the equations in the expected output.  

% 5.8.5  Preperation
\subsubsection{Preperation}
Create an objects and give them the initial values indicated in \ref{testRotation2}.
Output the objects' velocities right after the collision



%5.9 Test identifier 
\subsection{Rotation with damping}
% 5.9.1 Means of Control
\subsubsection{Means of Control}
Manual
% 5.9.2 Input
\subsubsection{Input}
Use the input stated in \ref{testRotation3}

% 5.9.3 Expected Output
\subsubsection{Expected Output}

$\omega(t) = -\frac{\pi}{6}(0.77)^{t} - \frac{\pi}{8}t$, 
$\phi(t) = \frac{\pi}{4} - \frac{\pi}{6}t(0.77)^{t} + \frac{\pi}{16}t^2$

% 5.9.4 Test  Procedures
\subsubsection{Procedure}
Record the orientation and the angular velcoity of the object as it rotates. Determine if the 
orientation and the angular velocitiy follow the equations in the expected output.  

% 5.9.5  Preperation
\subsubsection{Preperation}
Create an objects and give them the initial values indicated in \ref{testRotation3}.
Output the objects' velocities right after the collision
%%%%%%%%%%%%%%%%%%%%%%%%
%
%	6.) Unit Test Description 
%
%%%%%%%%%%%%%%%%%%%%%%%%
\section{Unit Test Description}


\subsection{Module Information}

\subsubsection{Module Inputs}

\subsubsection{Module Outputs}

\subsubsection{Related Modules}

\subsection{Test Data}

\subsubsection{Inputs}

\subsubsection{Expected Outputs}


\bibliographystyle {plain}
\bibliography {PCM_SRS}


%%%%%%%%%%%%%%%%%%%
%
%	Appendix
%
%%%%%%%%%%%%%%%%%%%
\section{Appendix}

\subsection{Test for Gravity} \label{testGravity}
The following test cases use:\\ \\
$ \textbf p_\text{i} = \begin{bmatrix}
	0\\
	50\\
	\end{bmatrix}$\\
~\newline 
\noindent \\
$ \textbf a = \begin{bmatrix}
	0\\
	-9.8\\
	\end{bmatrix}$\\
~\newline 
\noindent \\
t = 0, 0.5, 1, 1.5, 2.0, 2.5, 3.0, 3.5, 4.0

\subsubsection{No Initial Velocity} \label{testGravity1}
Use the values in \ref{testGravity} and an initial velocity of: \\ \\
$ \textbf v_\text{i} = \begin{bmatrix}
	0\\
	0\\
	\end{bmatrix}$\\
	
\subsubsection{Initial Velocity in y} \label{testGravity2}
Use the values in \ref{testGravity} and an initial velocity of: \\ \\
$ \textbf v_\text{i} = \begin{bmatrix} 
	0\\
	2.3\\
	\end{bmatrix}$\\
	
\subsubsection{Initial Velocity in x} \label{testGravity3}
Use the values in \ref{testGravity} and an initial velocity of: \\ \\
$ \textbf v_\text{i} = \begin{bmatrix}
	5.75\\
	0\\
	\end{bmatrix}$\\
	
\subsubsection{Initial Velocity in both x and y} \label{testGravity4}
Use the values in \ref{testGravity} and an initial velocity of: \\ \\
$ \textbf v_\text{i} = \begin{bmatrix}
	15.9\\
	4.8\\
	\end{bmatrix}$\\
	
	
\subsubsection{Parabolic Motion with Damping} \label{testGravity5}
Use the values in \ref{testGravity}, an initial velocity of: \\ \\
$ \textbf v_\text{i} = \begin{bmatrix}
	6.32\\
	-1.77\\
	\end{bmatrix}$\\
\noindent
and a damping coeffiecient: $\zeta = 0.95$


\subsection{Test for Collision} \label{testCollision}
The following test cases use:\\ \\
$ \textbf p_\text{i1} = \begin{bmatrix}
	100.5\\
	51.75\\
	\end{bmatrix}$\\
~\newline 
\noindent \\
$ \textbf p_\text{i2} = \begin{bmatrix}
	105\\
	50\\
	\end{bmatrix}$\\
~\newline 
\noindent\\
$ \textbf v_\text{i1} = \begin{bmatrix}
	3.1\\
	2.5\\
	\end{bmatrix}$\\
~\newline 
\noindent\\
$ \textbf v_\text{i2} = \begin{bmatrix}
	-5.4\\
	1.2\\
	\end{bmatrix}$\\
~\newline 
\noindent

\subsubsection{Conservation of momentum} \label{testCollision1}
Use the values in \ref{testCollision}, and $m_\text{1} = 5, m_\text{2} = 15.4, CR = 0.5$


\subsubsection{Perfectly elastic collision} \label{testCollision2}
Use the values in \ref{testCollision}, and  $m_\text{1} = 7.8, m_\text{2} = 20.1, CR = 1$
	
\subsection{Test for rotation} \label{testRotation}
The folllowing test cases use:

$ \textbf p_\text{i1} = \begin{bmatrix}
	0\\
	0\\
	\end{bmatrix}$\\
\noindent\\

$\phi = \frac{\pi}{4}$

\subsubsection{Rotation no initial angular velocity} \label{testRotation1}
Use the values in \ref{testRotation}, and $\omega = 0 $ and  $ \alpha = \frac{\pi}{10} $

\subsubsection{Rotation with angular velocity} \label{testRotation2}
Use the values in \ref{testRotation}, and $\omega = \frac{\pi}{3} $ and  $ \alpha = -\frac{\pi}{12} $

\subsubsection{Rotation with damping} \label{testRotation3}
Use the values in \ref{testRotation}, and $\omega = -\frac{\pi}{6} $,  $ \alpha = \frac{\pi}{8}$ and $\zeta = 0.77 $
\end{document}
\documentclass[12pt, titlepage]{article}

\usepackage{xcolor} % for different colour comments
\usepackage{tabto}
\usepackage{mdframed}
\usepackage{xkeyval}
\usepackage{tabularx}
\usepackage{booktabs}
\usepackage{hyperref}
\hypersetup{
    colorlinks,
    citecolor=black,
    filecolor=black,
    linkcolor=red,
    urlcolor=blue
}
\usepackage[singlelinecheck=off, skip=2pt, labelfont=bf]{caption}

%% Comments
\newif\ifcomments\commentstrue

\ifcomments
\newcommand{\authornote}[3]{\textcolor{#1}{[#3 ---#2]}}
\newcommand{\todo}[1]{\textcolor{red}{[TODO: #1]}}
\else
\newcommand{\authornote}[3]{}
\newcommand{\todo}[1]{}
\fi

\newcommand{\wss}[1]{\authornote{magenta}{SS}{#1}}
\newcommand{\ds}[1]{\authornote{blue}{DS}{#1}}

\makeatletter

\define@cmdkey      [SRS] {req}     {use}       {}
\define@cmdkey      [SRS] {req}     {desc}      {}
\define@cmdkey      [SRS] {req}     {rationale} {}
\define@cmdkey      [SRS] {req}     {orig}      {}
\define@cmdkey      [SRS] {req}     {fit}       {}
\define@cmdkey      [SRS] {req}     {sat}       {}
\define@cmdkey      [SRS] {req}     {dissat}    {}
\define@cmdkey      [SRS] {req}     {priority}  {}
\define@cmdkey      [SRS] {req}     {conf}      {}
\define@cmdkey      [SRS] {req}     {supp}      {}
\define@cmdkey      [SRS] {req}     {hist}      {}

\define@cmdkey      [SRS] {usecase} {name}      {}
\define@cmdkey      [SRS] {usecase} {trigger}   {}
\define@cmdkey      [SRS] {usecase} {precond}   {}
\define@cmdkey      [SRS] {usecase} {postcond}  {}
\define@cmdkey      [SRS] {usecase} {actor}     {}

\newcounter{UseCaseNum}
\newcommand{\usecase}[1]{
\setkeys[SRS]{usecase}{#1}
\refstepcounter{UseCaseNum}
\begin{mdframed}
\noindent {\bf Use Case \#:} \arabic{UseCaseNum}\\[\baselineskip]
\begin{tabularx}{\textwidth}{@{}p{3cm}X@{}}
{\bf Name:} & \cmdSRS@usecase@name\\
{\bf Trigger:} & \cmdSRS@usecase@trigger\\
{\bf Precondition:} & \cmdSRS@usecase@precond\\
{\bf Postcondition:} & \cmdSRS@usecase@postcond\\
{\bf Actor:} & \cmdSRS@usecase@actor\\[\baselineskip]
\end{tabularx}
\end{mdframed}
}


\newcounter{ReqNum}
\newcommand{\req}[1]{
\setkeys[SRS]{req}{#1}
\stepcounter{ReqNum}
\begin{mdframed}
\noindent {\bf Requirement \#:} \arabic{ReqNum} \tabto{4.6cm} {\bf Requirement Type:} \arabic{section}.\arabic{subsection} \tabto{10.2cm} {\bf Use Case:} \cmdSRS@req@use\\[\baselineskip]
\begin{tabularx}{\textwidth}{@{}p{3cm}X@{}}
{\bf Description:} & \cmdSRS@req@desc\\
{\bf Rationale:} & \cmdSRS@req@rationale\\
{\bf Fit Criterion:} & \cmdSRS@req@fit\\[\baselineskip]
\end{tabularx}
\begin{tabularx}{0.5\textwidth}{@{}p{4cm}X@{}}
{\bf Cust. Satisfaction:} & \cmdSRS@req@sat \\
{\bf Priority:} & \cmdSRS@req@priority \\[\baselineskip]
\end{tabularx}
\begin{tabularx}{0.5\textwidth}{@{}p{4.5cm}X@{}}
{\bf Cust. Dissatisfaction:} & \cmdSRS@req@dissat\\
{\bf Conflicts:} & \cmdSRS@req@conf\\[\baselineskip]
\end{tabularx}
\begin{tabularx}{\textwidth}{@{}p{5cm}X@{}}
{\bf Supporting Materials:} & \cmdSRS@req@supp\\
{\bf History:} & \cmdSRS@req@hist
\end{tabularx}
\end{mdframed}}

\makeatother


\begin{document}

\title{\bf Physics-Based Chipmunk2D Game\\[\baselineskip]\Large Software Requirements Specification\\[2\baselineskip] \large Based on the Volere Template}
\author{Steven Palmer\\Emaad Fazal\\Chao Ye}
\date{\today}
	
\maketitle

\pagenumbering{roman}
\tableofcontents
\listoftables

\begin{table}[bp]
\caption*{\bf Revision History}
\begin{tabularx}{\textwidth}{p{3cm}p{2cm}X}
\toprule {\bf Date} & {\bf Version} & {\bf Notes}\\
\midrule
October 7, 2015 & 1.0 & Created document\\
October 7, 2015 & 1.1 & Major edits in progress\\
\bottomrule
\end{tabularx}
\end{table}

\newpage
\pagenumbering{arabic}
\section{Project Drivers}
\subsection{The Purpose of the Project}
The purpose of this project is to produce a game that will be used as a demonstration for students in a third year software engineering game design course at McMaster University.  The game will incorporate the \href{https://chipmunk-physics.net/}{Chipmunk2D} physics library and highlight its capabilities.
\subsection{The Stakeholders}
\subsubsection{The Client}
The client for this project is \href{http://www.cas.mcmaster.ca/~smiths/}{Dr.~Spencer Smith} of the Computing and Software department at McMaster University.
\subsubsection{The Customer}
The customer for this project are students who will take the game design course in the future.
\subsubsection{Other Stakeholders}
Other stakeholders include future instructors of the game design course and other related courses.
\section{Project Description}
\subsection{Game Overview}
The game will follow along those lines allowing the user to play as his or her own character through multiple levels where the user levels up the more and better he plays. The user will be able to use multiple weapons to combat a variety of enemies.  Experience is accumulated by defeating enemies and is used to progress the hero. As the user’s character progresses through the game it will become more difficult to level up as enemy difficulty is substantially increased.

The game consists of the game world, within which the hero and interact with enemies.

\subsubsection{Game World}
The maps, otherwise known as levels, provide an interactive way for the user to progress deeper into the game. Some maps have a beginning and end, like a 2-D Mario level, whereas certain levels are constant in their placement, however have more enemies.
 
The game world consists of a virtual environment in which the gameplay takes place.  This environment is made up of platforms which control where the hero and enemies are permitted to, as well as blockades which limit the possible movements of the hero throughout the game.   The maps have environmental hazards and objects which the user is able to interact with. Some objects, such as power ups, give the user temporary boosts while other map elements such as spikes and fire can cause damage to the user. Besides these two the user is also responsible for navigating around blockades placed on the level. This provides a challenge for the user as he navigates through the level.


\subsubsection{The Hero}
The hero is the protagonist of the game, and is controlled by the user.  The hero is able to move left or right and to jump in order to progress through the game.  The hero can interact with several objects in the game, such as enemies, blockades, and hazards.  When the hero comes into contact with an object an event is triggered.  Depending on the type of object, events include:

\begin{enumerate}
  \item If the object is an enemy the hero will lose health and be knocked back from the enemy.
  \item If the object in a blockade the hero will be stopped and unable to pass.
  \item If the object is an environmental hazard the hero will lose health and may be knock back depending on the type of hazard.  Spikes, for example, will cause a knock-back effect, whereas fire would not.
  \item If the object is an item the hero will gain some bonus or ability.  
\end{enumerate}

%  WE CAN PROBABLY LEAVE OUT THE WEAPON SYSTEM AT THIS POINT

%The last important aspect of characters is how he uses weapons. The character will use weapons to defeat enemies. The character can cycle through weapons choosing his preferred weapon of choice.


\subsubsection{The Enemies}
	When an enemy is encountered during the game, combat may ensue if the user wishes to attack the enemy.  The use may also attempt to avoid the enemy altogether. Users can employ multiple different weapons to attack enemies. Enemies in their own capacity are capable of engaging in contact with the players. The enemies may use weapon or just come in contact with the player to instantiate damage. If the enemy weapons or the enemy itself come in contact with the user then the player loses health relative to the type of enemy it is. Different enemies deal different levels of damage. The stronger enemies may deal more damage, however are slower, whereas the weaker ones attack at a higher frequency but with less damage. The goal of the enemies is to completely make the users health reach zero.

Maps:

%Weapon functionality :
%Weapons provide a way for users and enemies to interact in combat. The user has certain weapons which are given to him as artillery, however there are environmental weapons that the user and enemies may use. Whomever reaches the environmental weapon first will get the weapon. Weapons have certain characteristics that allow the user to choose which weapon will vest suit the situation at hand. Some weapons have a higher fire rate which allows the user to attack a horde of enemies at once. The trade-off is the damage of these weapons will be lower. Certain weapons fire in a projectile range so the user can attack enemies waiting around blockades. And certain weapons have high damage, but the trade-off is these weapons have a low fire rate or long reload time. Overall, the types of weapons vary in qualities that quality of their usage depending on the situation. If ten enemies surround you may want to use a high fire weapon, however if less enemies are involved you may want to use more damage to get rid of them quickly


\subsection{Mandated Constraints}

The project is subject to the following mandated constraints:

\begin{enumerate}
  \item The game must make significant use of the Chipmunk2D physics library.
  \item The game must support all major PC operating systems.
  \item Project milestones must be completed by the dates given in the CS 4ZP6 syllabus.
  \item The project must be fully completed by April 1, 2015.
\end{enumerate}

\subsection{Naming Conventions and Terminology}
The terminology used in this project is given in \hyperref[tab:terminology]{Table~\ref*{tab:terminology}}.
\begin{table}
\caption{List of terminology} \label{tab:terminology}
\begin{tabularx}{\textwidth}{p{3cm}X}
\toprule {\bf Term} & {\bf Definition}\\
\midrule
Boss & Important enemy with increased stats\\
Bounds & The boundaries inside which game play occurs\\
Character & A\\
Enemy & Hostile NPC; normally attacks hero\\
Hero & The main character of the game controlled by the user\\
NPC & Non-playable character; may be friendly or hostile\\
\bottomrule
\end{tabularx}
\end{table}

\section{Functional Requirements}
\subsection{The Scope of the Work}

\usecase{
    name = Exit Game,
    trigger = The user selects exit game,
    precond = The main menu is open,
    postcond = The application is terminated,
    actor = User
} \label{usecase:exitgame}

\usecase{
    name = New Game,
    trigger = The user selects to start a new game,
    precond = The main menu is open,
    postcond = A new game commences,
    actor = User
}

\usecase{
    name = Load Game,
    trigger = The user selects to load a game,
    precond = The main menu or in-game menu is open,
    postcond = A saved game state is loaded and the game commences from that point,
    actor = User
}

\usecase{
    name = Save Game,
    trigger = The user selects to save a game,
    precond = The in-game menu is open,
    postcond = A saved game state is created,
    actor = User
}

\usecase{
    name = Move Hero,
    trigger = Inputs related to controlling the hero,
    precond = The in-game menu is open,
    postcond = A saved game state is created,
    actor = User
}

\usecase{
    name = Combat,
    trigger = Hero comes in contact with enemy,
    precond = 
    postcond = 
    actor = User
} 

\usecase{
    name = Open In-game Menu,
    trigger = The user ,
    precond = The user is currently in game,
    postcond = The in-game menu is opened,
    actor = User
}

\usecase{
    name = Open Hero Menu,
    trigger = The  ,
    precond = The user is currently in game,
    postcond = The in-game menu is opened,
    actor = User
}

\subsection{The Scope of the Product}



\subsection{Functional Requirements}

\req{
        use = \ref{usecase:exitgame},
        desc = The \hyperref[tab:terminology]{user} shall have the ability to load a saved game state,
        rationale = The \hyperref[tab:terminology]{user} must be able to save his or her progress,
        orig = Steven Palmer,
        fit = \hyperref[tab:terminology]{User} is able to successfully load game,
        sat = 1,
        dissat = 5,
        priority = High,
        conf = None,
        supp = None,
        hist = {Created October 7, 2015}
}

\req{
        use = 1,
        desc = The \hyperref[tab:terminology]{hero} shall remain in \hyperref[tab:terminology]{bounds},
        rationale = The \hyperref[tab:terminology]{hero} must remain in the intended boundaries of play for the game to function properly,
        orig = Steven Palmer,
        fit = \hyperref[tab:terminology]{Hero} is unable to pass through walls and other obstacles,
        sat = 5,
        dissat = 5,
        priority = High,
        conf = None,
        supp = None,
        hist = {Created October 7, 2015}
}

\section{Non-functional Requirements}
\subsection{Look and Feel Requirements}
\subsection{Usability and Humanity Requirements}
\req{
        use = 1,
        desc = The game shall be entertaining,
        rationale = A game that is not fun is a failure,
        orig = Steven,
        fit = The game should be ranked at least 7/10 for entertainment based on a usability study,
        sat = 2,
        dissat = 5,
        priority = High,
        conf = None,
        supp = None,
        hist = {Created October 7, 2015}
}
\subsection{Performance Requirements}
\req{
        use = 1,
        desc = The game shall maintain a framerate of at least 30 fps,
        rationale = A framerate of 30 fps or greater will ensure smooth animation,
        orig = Steven,
        fit = The game shall ,
        sat = 2,
        dissat = 5,
        priority = High,
        conf = None,
        supp = None,
        hist = {Created October 7, 2015}
}
\subsection{Operational and Environmental Requirements}
\subsection{Maintainability and Support Requirements}
\req{
        use = 1,
        desc = {The game shall support Windows, Linux, and OS X operating systems},
        rationale = Students use a variety of operating systems,
        orig = Spencer Smith,
        fit = Check that game compiles and runs on each operating system,
        sat = 3,
        dissat = 3,
        priority = High,
        conf = None,
        supp = None,
        hist = {Created October 7, 2015}
}
\subsection{Security Requirements}
There are no security requirements related to this project.
\subsection{Cultural Requirements}
\req{
        use = N/A,
        desc = {The game shall use the English language},
        rationale = Students at McMaster University are expected to speak English,
        orig = Steven,
        fit = The game will use proper English free of spelling and grammar errors,
        sat = 1,
        dissat = 3,
        priority = Medium,
        conf = None,
        supp = None,
        hist = {Created October 7, 2015}
}
\subsection{Legal Requirements}
There are no legal requirements related to this project.

\section{Project Issues}
\subsection{Open Issues}
There are no open issues at this time.  This section will be updated as required.
\subsection{Off-the-Shelf Solutions}
There are no off-the-shelf solutions.
\subsection{New Problems}
No new problems are expected to arise as a result of this project.
\subsection{Tasks}

\subsection{Migration to the New Product}
There is no product being replaced, and thus no migration is required.
\subsection{Risks}
There are no risks associated with this project.
\subsection{Costs}
There are no costs associated with this project.
\subsection{User Documentation and Training}
User documentation will be created with .  Training will be provided via built-in tutorials throughout the game.
\subsection{Waiting Room}
At this point in the project timeline, there are no backlogged requirements.  This section will be updated as required.
\subsection{Ideas for Solutions}
There are no ideas for solutions at this time.  This section will be updated as required.
\end{document}
